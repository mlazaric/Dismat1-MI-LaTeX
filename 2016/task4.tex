\documentclass[exam.tex]{subfiles}

\begin{document}
	\begin{task}
		Dokaži identitet
	
		\[ \binom{n}{0} + 3 \binom{n}{1} + 5 \binom{n}{2} + 7 \binom{n}{3} + \cdots = 2^n (n + 1) \] \\[1em]
	\end{task}

	\textbf{1. način} (ideja je predložio \UkiseljeniKrastavac)
	
	Započnemo od formule za rastav binoma:
	
	\[ (x + 1)^n = \binom{n}{0} + \binom{n}{1} x + \binom{n}{2} x^2 + \cdots + \binom{n}{n} x^n \]
	
	Deriviramo tu jednadžbu.
	
	\[ n (x + 1)^{n - 1} = \binom{n}{1} + 2 \binom{n}{2} x + \cdots + n \binom{n}{n} x^{n - 1} \]
	
	Pomnožimo dobivenu jednadžbu s \( 2 \).
	
	\[ 2 n (x + 1)^{n - 1} = 2 \binom{n}{1} + 4 \binom{n}{2} x + \cdots + 2 n \binom{n}{n} x^{n - 1} \]
	
	Zbrojimo prvu i zadnju jednadžbu.
	
	\[ (x + 1)^n + 2 n (x + 1)^{n - 1} = \binom{n}{0} + 3 \binom{n}{1} x + 5 \binom{n}{2} x^2 + \cdots + (n + 1) \binom{n}{n} x^n \]
	
	Uvrstimo \( x = 1 \).
	
	\[ 2^n + 2 n \cdot 2^{n - 1} = \binom{n}{0} + 3 \binom{n}{1} + 5 \binom{n}{2} + \cdots + (n + 1) \binom{n}{n} \]
	
	Faktoriziramo lijevu stranu i dokazali smo identitet.
	
	\[ 2^n (n + 1) = \binom{n}{0} + 3 \binom{n}{1} + 5 \binom{n}{2} + \cdots + (n + 1) \binom{n}{n} \] 
	
	\textbf{2. način}
	
	Znamo da će svaki \( \displaystyle \binom{n}{k} \) za koji vrijedi \( k > n \) biti jednak 0. To proizlazi iz formule za binomni koeficijent \( \displaystyle \binom{n}{k} = \frac{n^{\underline{k}}}{k!} \) gdje je \( n^{\underline{k}} = n (n - 1) \cdots (n - k + 1) \), pa za \( k > n\) množimo s nulom. Onda možemo smanjiti beskonačnu sumu na konačnu.
	
	\[ \binom{n}{0} + 3 \binom{n}{1} + \cdots + (2n - 1) \binom{n}{n - 1} + (2n + 1) \binom{n}{n} = 2^n (n + 1) \]
	
	Grupiramo prvi i zadnji, drugi i predzadnji itd. jer vrijedi \( \binom{n}{k} = \binom{n}{n - k} \). \\
	
	Ovdje je važno razmatrati kakva je situacija za parne \( n \)-ove, a kakva za neparne. Za \textbf{parne} \( n = 2k \) će vrijediti:
	
	\[ 2(n + 1) \binom{n}{0} + 2(n + 1) \binom{n}{1} + \cdots + 2(n + 1) \binom{n}{k - 1} + (n + 1) \binom{n}{k} = 2^n (n + 1) \]
	
	odnosno svi binomni koeficijenti imaju svojeg para \textbf{osim centralnog binomnog koeficijenta} \( \binom{n}{k} \). \\
	
	Za \textbf{neparne} \( n = 2k + 1 \) će vrijediti:
	
	\[ 2(n + 1) \binom{n}{0} + 2(n + 1) \binom{n}{1} + \cdots 
	+ 2(n + 1) \binom{n}{k - 1}  + 2(n + 1) \binom{n}{k} = 2^n (n + 1) \]
	
	odnosno svi binomni koeficijenti imaju svojeg para. \\
	
	Iako je situacija različita za parne i neparne \( n \)-ove, kada ih rastavimo natrag tako da svakog množi \( n + 1 \) situacija će biti jednaka. U slučaju neparnih samo ih prepolovimo, dok u slučaju parnih prepolovimo sve osim centralnog kojeg ostavimo takvog kakav je. \\
	
	Rastavimo ih po istom svojstvu na navedeni način, tako da možemo faktorizirati \( (n + 1) \).
	\begin{align*}
		(n + 1) \binom{n}{0} + (n + 1) \binom{n}{1} + \cdots + (n + 1) \binom{n}{n} &= 2^n (n + 1) \\
		(n + 1) \left (\binom{n}{0} + \binom{n}{1} + \cdots + \binom{n}{n} \right) &= 2^n (n + 1) \\
		(n + 1) \sum\limits^n_{k=0} \binom{n}{k} &= 2^n (n + 1) \\
		(n + 1) 2^n &= 2^n (n + 1)
	\end{align*}
	
	Koristili smo svojstvo \( \sum\limits^n_{k=0} \binom{n}{k} = 2^n \). 
\end{document}