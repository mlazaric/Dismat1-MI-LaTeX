\documentclass{article}

\usepackage[croatian]{babel}
\usepackage[utf8]{inputenc}
\usepackage{amsmath}

\begin{document}
	Dokaži identitet
	
	\[ \binom{n}{0} + 3 \binom{n}{1} + 5 \binom{n}{2} + 7 \binom{n}{3} + \cdots = 2^n (n + 1) \]
	
	Znamo da će svaki \( \displaystyle \binom{n}{k} \) za koji vrijedi \( k > n \) biti jednak 0. To proizlazi iz formule za binomni koeficijent \( \displaystyle \binom{n}{k} = \frac{n^{\underline{k}}}{k!} \) gdje je \( n^{\underline{k}} = n (n - 1) \cdots (n - k + 1) \), pa za \( k > n\) množimo s nulom. Onda možemo smanjiti beskonačnu sumu na konačnu:
	
	\[ \binom{n}{0} + 3 \binom{n}{1} + \cdots + (2n - 1) \binom{n}{n - 1} + (2n + 1) \binom{n}{n} = 2^n (n + 1) \]
	
	Grupiramo prvi i zadnji, drugi i predzadnji itd. jer vrijedi \( \binom{n}{k} = \binom{n}{n - k} \)
	
	\[ 2(n + 1) \binom{n}{0} + 2(n + 1) \binom{n}{1} + \cdots + 2(n + 1) \binom{n}{\frac{n}{2}} = 2^n (n + 1) \]
	
	Rastavimo ih po istom svojstvu tako da možemo faktorizirati \( (n + 1) \).
	
	\begin{align*}
		(n + 1) \binom{n}{0} + (n + 1) \binom{n}{1} + \cdots + (n + 1) \binom{n}{n} &= 2^n (n + 1) \\
		(n + 1) \left (\binom{n}{0} + \binom{n}{1} + \cdots + \binom{n}{n} \right) &= 2^n (n + 1) \\
		(n + 1) \sum\limits^n_{k=0} \binom{n}{k} &= 2^n (n + 1) \\
		(n + 1) 2^n &= 2^n (n + 1)
	\end{align*}
	
	Koristili smo svojstvo \( \sum\limits^n_{k=0} \binom{n}{k} = 2^n \).
\end{document}