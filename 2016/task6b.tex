\documentclass[exam.tex]{subfiles}

\begin{document}
	Izvedi formulu za funkciju izvodnicu
	
	b) Fibonaccijevih brojeva
	
	Koristimo temeljnu relaciju Fibonaccijevih brojeva:
	
	\[ F_n = F_{n - 1} + F_{n - 2} \]
	
	To riješimo standardnim postupkom za rješavanje rekurzije preko funkcija izvodnica, prvo možemo sve prenijeti na lijevu stranu.
	
	\[ F_n - F_{n - 1} - F_{n - 2} = 0 \]
	
	Onda odredimo funkciju izvodnicu za svaki član. \\
	
	Prvi član nam je funkcija izvodnica Fibonaccijevih brojeva: 
	\[ F(x) = F_0 + F_1 x + F_2 x^2 + \cdots \]
	
	Drugi član nam je funkcija izvodnica Fibonaccijevih brojeva pomaknuta za jedan i s negativnim predznakom:
	\[ - x F(x) = - F_0 x - F_1 x^2 - F_2 x^3 - \cdots \]
	
	Treći član nam je funkcija izvodnica Fibonaccijevih brojeva pomaknuta za dva i s negativnim predznakom:
	\[ - x^2 F(x) = - F_0 x^2 - F_1 x^3 - F_2 x^4 - \cdots \]
	
	Onda zbrojimo te tri jednadžbe i grupiramo članove po potencijama x.
	\[ F(x) - x F(x) - x^2 F(x) = F_0 + x (F_1 - F_0) + x^2 (F_2 - F_1 - F_0) + \cdots \]
	
	Od \( n = 2 \) nadalje će vrijediti \[ F_n - F_{n - 1} - F_{n - 2} = 0 \] pa nam preostaju slobodni član i prva potencija:
	\[ F(x) - x F(x) - x^2 F(x) = F_0 + x (F_1 - F_0) \]
	
	Uvrstimo \( F_0 = 0 \) i \( F_1 = 1 \), te faktoriziramo lijevu stranu.
	\[ F(x)(1 - x - x^2) = x \]
	
	Podijelimo i dobili smo rješenje:
	\[ F(x) = \frac{x}{1 - x - x^2} \]
\end{document}