\documentclass[exam.tex]{subfiles}

\begin{document}
	\begin{subtask}
		Izvedi formulu za funkciju izvodnicu Catalanovih brojeva \\
	\end{subtask}
	
	Ako definiramo funkciju izvodnicu za Catalanove brojeve kao:
	
	\[ C(x) = \sum\limits^\infty_{n=0} x^n C_n \]
	
	Onda izdvojimo \( C_0 \) iz sume.
	
	\[ C(x) = C_0 + 	\sum\limits^\infty_{n=1} x^n C_n \]
	
	Pomaknemo indeks sumacije i prebacimo \( C_0 \) na lijevu stranu.
	
	\[ C(x) - C_0 = 	\sum\limits^\infty_{n=0} x^{n + 1} C_{n + 1} \]
	
	Tada uvrstimo osnovnu relaciju Catalanovih brojeva \( C_{n + 1} = \sum\limits^n_{k=0} C_k C_{n - k} \).
	
	\[ C(x) - C_0 = 	\sum\limits^\infty_{n=0} x^{n + 1} \sum\limits^n_{k=0} C_k C_{n - k} \]
	
	Na desnoj strani izvadimo \( x \) ispred sume kako bismo imali \( x^n \).
	
	\[ C(x) - C_0 = 	x \sum\limits^\infty_{n=0} x^n \sum\limits^n_{k=0} C_k C_{n - k} \]
	
	Uočimo da je na desnoj strani Cauchyjev produkt redova koji glasi \( A(x) B(x) = \sum\limits^\infty_{n=0} x^n \sum\limits^n_{k=0} a_k b_{n - k} \), a u našem slučaju su oba reda \( C(x) \).
	
	\[ C(x) - C_0 = 	x C(x)^2 \]
	
	Prebacimo sve na desnu stranu i uvrstimo \( C_0 = 1 \).
	
	\[ 0 = 	x C(x)^2 - C(x) + 1 \]
	
	Riješimo kvadratnu jednadžbu i dobijemo dva rješenja.
	
	\[ C(x) = \frac{1}{2x} \left ( 1 \pm \sqrt{1 - 4x} \right ) \]
	
	Odaberemo \( - \), zato što za \( + \) funkcija teži k \( + \infty \) kada x teži k \( 0 \), pa je konačno rješenje:
	
	\[ C(x) = \frac{1}{2x} \left ( 1 - \sqrt{1 - 4x} \right ) \]
\end{document}