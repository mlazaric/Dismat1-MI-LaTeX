\documentclass[exam.tex]{subfiles}

\begin{document}
	\begin{task}
		Zamjenom poretka sumacije, izračunaj sumu \( \displaystyle \sum\limits^n_{k=0} \sum\limits^n_{j=k} \binom{n}{j} \binom{j}{k} \) \\[1em]
	\end{task}
	
	Zamijenimo poredak sumacije.
	
	\[ \sum\limits^n_{j=0} \sum\limits^j_{k=0} \binom{n}{j} \binom{j}{k} \]
	
	Izvadimo prvi binomni koeficijent izvan unutarnje sume jer ne ovisi o \( k \).
	
	\[ \sum\limits^n_{j=0} \binom{n}{j} \sum\limits^j_{k=0} \binom{j}{k} \]
	
	Vrijedi \( \displaystyle \sum\limits^n_{k=0} \binom{n}{k} = 2^n \), pa možemo riješiti unutarnju sumu.
	
	\[ \sum\limits^n_{j=0} \binom{n}{j} 2^j \]
	
	Također raspis binoma glasi \( \displaystyle (x + 1)^n = \sum\limits^n_{k=0} \binom{n}{k} x^n \). Ako uvrstimo \( x = 2 \) dobijemo istu sumu pa vrijedi:
	
	\[ \sum\limits^n_{j=0} \binom{n}{j} 2^j = (2 + 1)^n = 3^n \]
\end{document}