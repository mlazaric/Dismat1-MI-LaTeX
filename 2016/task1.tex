\documentclass[exam.tex]{subfiles}

\begin{document}
	\begin{task}
		Izračunaj sumu
	
		\[ S_n = \frac{3}{2} + \frac{4}{2^2} + \frac{5}{2^3} + \cdots + \frac{n+2}{2^n} \]
	\end{task}
	
	Pretpostavimo rješenje oblika:
	
	\[ S_n = r + q^n P(n) \]
	
	gdje je \( P(n) = an + b \) polinom prvog stupnja kao i opći član početne sume, a \( q = \frac{1}{2} \).
	
	Onda vrijedi:
	
	\[ S_n = r + 2^{-n} (an + b) \]
	
	Odnosno za \( S_{n+1} \) vrijedi:
	
	\[ S_{n + 1} = r + 2^{-n-1} (a(n + 1) + b) \]
	
	Također vrijedi i
	
	\begin{align*}
		S_{n + 1} &= S_n + a_{n + 1} \\
		&= S_n + 2^{-n - 1} (n + 1 + 2)  \\
		&= r + 2^{-n} (an + b) + 2^{-n-1} (n + 3) 
	\end{align*}
	
	Izjednačimo jednadžbe
	
	\[ r + 2^{-n-1} (a(n + 1) + b) = r + 2^{-n} (an + b) + 2^{-n-1} (n + 3) \]
	
	Pokratimo r
	
	\[ 2^{-n-1} (a(n + 1) + b) = 2^{-n} (an + b) + 2^{-n-1} (n + 3) \]
	
	Pomnožimo s \( 2^{n+1} \)
	
	\[ a(n + 1) + b = 2 (an + b) +  n + 3 \]
	
	Iz toga slijedi sustav jednadžbi
	
	\begin{align*}
		a &= 2a + 1 \\
		a + b &= 2b + 3
	\end{align*}
	
	Pa dobijemo 
	
	\begin{align*}
		a &= -1 \\
		b &= -4
	\end{align*}
	
	To uvrstimo natrag u formulu za \( S_n \) i izračunamo \( S_1 \) kako bismo izračunali \( r \).
	
	\begin{align*}
		S_1 &= \frac{3}{2} \\ 
		\\
		\frac{3}{2} &= r + \frac{-1 \cdot 1 - 4}{2} \\
		\\	
		r &= 4
	\end{align*}
	
	Konačno rješenje je:
	
	\[ S_n = 4 + 2^{-n} (-n -4) \]
\end{document}