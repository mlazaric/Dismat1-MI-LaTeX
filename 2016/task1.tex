\documentclass[exam.tex]{subfiles}

\begin{document}
	\begin{task}
		Izračunaj sumu
	
		\[ S_n = \frac{3}{2} + \frac{4}{2^2} + \frac{5}{2^3} + \cdots + \frac{n+2}{2^n} \]\\[1em]
	\end{task} 
	
	\textbf{1. način: univerzalna metoda}
	
	Pretpostavimo rješenje oblika:
	
	\[ S_n = r + q^n P(n) \]
	
	gdje je \( P(n) = an + b \) polinom prvog stupnja kao i opći član početne sume, a \( q = \frac{1}{2} \). \\
	
	Onda vrijedi:
	
	\[ S_n = r + 2^{-n} (an + b) \]
	
	Odnosno za \( S_{n+1} \) vrijedi:
	
	\[ S_{n + 1} = r + 2^{-n-1} (a(n + 1) + b) \]
	
	Također vrijedi i:
	\begin{align*}
		S_{n + 1} &= S_n + a_{n + 1} \\
		&= S_n + 2^{-n - 1} (n + 1 + 2)  \\
		&= r + 2^{-n} (an + b) + 2^{-n-1} (n + 3) 
	\end{align*}
	
	Izjednačimo jednadžbe.
	
	\[ r + 2^{-n-1} (a(n + 1) + b) = r + 2^{-n} (an + b) + 2^{-n-1} (n + 3) \] 
	
	Pokratimo \( r \).
	
	\[ 2^{-n-1} (a(n + 1) + b) = 2^{-n} (an + b) + 2^{-n-1} (n + 3) \]
	
	Pomnožimo s \( 2^{n+1} \).
	
	\[ a(n + 1) + b = 2 (an + b) +  n + 3 \]
	
	Iz toga slijedi sustav jednadžbi:
	\begin{align*}
		a &= 2a + 1 \\
		a + b &= 2b + 3
	\end{align*}
	
	Pa dobijemo:
	\begin{align*}
		a &= -1 \\
		b &= -4
	\end{align*}
	
	To uvrstimo natrag u formulu za \( S_n \) i izračunamo \( S_1 \) kako bismo izračunali \( r \).
	\begin{align*}
		S_1 &= \frac{3}{2} \\ 
		\frac{3}{2} &= r + \frac{-1 \cdot 1 - 4}{2} \\
		r &= 4
	\end{align*}
	
	Konačno rješenje je:
	
	\[ S_n = 4 + 2^{-n} (-n -4) \]
	
	\newpage
	
	\textbf{2. način: deriviranje} (ideju je predložio \UkiseljeniKrastavac)
	
	\begin{align*}
		S_n &= \frac{3}{2} + \frac{4}{2^2} + \frac{5}{2^3} + \cdots + \frac{n+2}{2^n}  \\
		&= \sum\limits_{k=1}^n \frac{k + 2}{2^k}
	\end{align*}
	
	Definiramo novu sumu.
	
	\begin{align*}
		A_n &= x^3 + x^4 + x^5 + \cdots + x^{n + 2} \\
		&= \sum\limits_{k=1}^n x^{k + 2}
	\end{align*}
	
	Primijetimo da smo odabrali \( \frac{1}{x^3} \) kao početni član kako bismo deriviranjem dobili \( 3 \) u brojniku. Isto vrijedi za izbor zadnjeg člana, odabrali smo ga kako bismo dobili \( n + 2 \) u brojniku.
	
	To je upravo geometrijski niz za koji znamo sumu.
	
	\[ A_n = x^3 \frac{1 - x^n}{1 - x} = \frac{x^3 - x^{n + 3}}{1 - x} \]
	
	Deriviramo.
	
	\begin{align*}
		\frac{d}{\mathop{dx}} \left ( \sum\limits_{k=1}^n x^{k + 2} \right ) &= \frac{d}{\mathop{dx}} \left ( \frac{x^3 - x^{n + 3}}{1 - x} \right ) \\
		\sum\limits_{k=1}^n (k + 2) x^{k + 1} &= \frac{(3x^2 - (n + 3)x^{n + 2})(1 - x) + x^3 - x^{n + 3}}{(1 - x)^2}  \\
		&= \frac{3x^2 - (n + 3)x^{n + 2} - 3x^3 + (n + 3)x^{n + 3} + x^3 - x^{n + 3}}{(1 - x)^2}  \\
		&= \frac{3x^2 - (n + 3)x^{n + 2} - 2x^3 + (n + 2)x^{n + 3}}{(1 - x)^2}  \\
		&= \frac{3x^2 - 2x^3 - (n + 3)x^{n + 2} + (n + 2)x^{n + 3}}{(1 - x)^2} 
	\end{align*}
	
	No, mi želimo da nam suma počinje od člana \( x \), a trenutna suma počinje od \( x^2 \), stoga pomnožimo s \( \frac{1}{x} \).
	
	\begin{align*}
		\sum\limits_{k=1}^n (k + 2) x^k	&= \frac{3x - 2x^2 - (n + 3)x^{n + 1} + (n + 2)x^{n + 2}}{(1 - x)^2} 
	\end{align*}
	
	Jedino što nam je preostalo je uvrstiti \( x = \frac{1}{2} \) i malo srediti.
	
	\begin{align*}
		\sum\limits_{k=1}^n \frac{k + 2}{2^k} &= \frac{\frac{3}{2} - \frac{2}{4} -  \frac{n + 3}{2^{n + 1}} + \frac{n + 2}{2^{n + 2}}}{\frac{1}{4}} \\
		&= 4 \frac{3 \cdot 2^{n + 1} - 2 \cdot 2^{n} - 2(n + 3) + n + 2}{2^{n + 2}} \\
		&= 4 \frac{4 \cdot 2^{n} - 2n - 6 + n + 2}{2^{n + 2}} \\
		&= 4 \frac{\cdot 2^{n+2} - n - 4}{2^{n + 2}} \\
		&= 4 - \frac{4(n + 4)}{2^{n + 2}} \\
		&= 4 - \frac{n + 4}{2^n}
	\end{align*}
\end{document}